% resume.tex
% vim:set ft=tex spell:

\documentclass[10pt,letterpaper]{article}
\usepackage[letterpaper,margin=0.75in]{geometry}
\usepackage[utf8]{inputenc}
\usepackage{mdwlist}
\usepackage[T1]{fontenc}
\usepackage{textcomp}
\usepackage{tgpagella}
\pagestyle{empty}
\setlength{\tabcolsep}{0em}

% indentsection style, used for sections that aren't already in lists
% that need indentation to the level of all text in the document
\newenvironment{indentsection}[1]%
{\begin{list}{}%
	{\setlength{\leftmargin}{#1}}%
	\item[]%
}
{\end{list}}

% opposite of above; bump a section back toward the left margin
\newenvironment{unindentsection}[1]%
{\begin{list}{}%
	{\setlength{\leftmargin}{-0.5#1}}%
	\item[]%
}
{\end{list}}

% format two pieces of text, one left aligned and one right aligned
\newcommand{\headerrow}[2]
{\begin{tabular*}{\linewidth}{l@{\extracolsep{\fill}}r}
	#1 &
	#2 \\
\end{tabular*}}

% and the actual content starts here
\begin{document}

\begin{center}
{\LARGE \textbf{Malcolm D. Reid Jr.}}

2308 University Ave. \ \ \textbullet
\ \ Apt.\ 30\ \ \textbullet
\ \ Madison, WI 53726
\\
(608) 361-8665\ \ \textbullet
\ \ mreid3@wisc.edu \ 
\end{center}

\hrule
\vspace{-0.4em}
\subsection*{Education}

\begin{itemize}
	\parskip=0.1em

	\item 
	\headerrow
		{\textbf{University of Wisconsin - Madison}}
		{\textbf{Madison, WI}}
	\\
	\headerrow
		{\emph{M.S. Computer Sciences}}
		{\emph{2015 -- May 2017 (Expected)}}
	\begin{itemize*}
        \item GPA: 4.00/4.00
		\item Relevant coursework includes Advanced Computer Networks (CS 740), Advanced Operating Systems (CS 736), and Advanced Algorithms (CS 787).
	\end{itemize*}
    \item 
	\headerrow
		{\textbf{Princeton University}}
		{\textbf{Princeton, NJ}}
	\\
	\headerrow
		{\emph{B.A. Psychology, with minors in Computer Science and Neuroscience}}
		{\emph{2010 -- 2014}}
	\begin{itemize*}
        \item GPA: 3.44/4.00
		\item Awarded Shapiro Award for Academic Excellence for the 2011-2012 academic year. This award is given to the top 3 \% of high-achieving freshmen and sophomores at Princeton.
		\item Relevant coursework includes Artificial Intelligence (COS 402), Reasoning about Computation (COS 340), and Advanced Programming Techniques (COS 333).
	\end{itemize*}

\end{itemize}

\hrule
\vspace{-0.4em}
\subsection*{Experience}

\begin{itemize}
	\parskip=0.1em

	\item
    \headerrow
		{\textbf{UW-Madison Software Assurance Marketplace}}
		{\textbf{Madison, WI}}
	\\
	\headerrow
		{\emph{Research Assistant under Dr. Bart Miller}}
		{\emph{2016 -- Present}}

		\begin{itemize*}
        \item Worked on Department of Homeland Security funded project (see https://continuousassurance.org/ for more details)
		\item Built IDE-specific plugins to allow application developers to have their code statically analyzed by tools on the Software Assurance Marketplace		
       \end{itemize*}

	\item
    \headerrow
		{\textbf{Epic Systems Corporation}}
		{\textbf{Verona, WI}}
	\\
	\headerrow
		{\emph{Software Developer, Application Support Engineer}}
		{\emph{2014 -- 2016}}

		\begin{itemize*}
        \item Worked on the anesthesia module of Epic's electronic medical record software to develop software for anesthesiologists and Certified Registered Nurse Anesthetists
        \item Focused on data visualization of patient vitals and medication administrations. Migrated anesthesia's chief visualization from the client to the server, thus allowing batch close encounters
        \item Member of the Infrastructure pod, which deals with infrastructural elements of the anesthesia module including chart access and security
        \item Developed extensive additions and modifications to Epic's anesthesia software to support analgesia workflows
        \item Served leadership roles including Technical Operations representative and Thin Logs expert
	    \end{itemize*}

    \item
    \headerrow
        {\textbf{Princeton University}}
        {\textbf{Princeton, NJ}}
    \\
    \headerrow
        {\emph{Research Assistant}}
        {\emph{2013 -- 2014}}

        \begin{itemize*}
        \item Worked in Computational Memory Lab at Princeton University from May 2013 to May 2014
        \item Trained machine learning classifiers on functional magnetic resonance imaging data to infer mental state
        \end{itemize*}
    
    \item
    \headerrow
        {\textbf{Brigham and Women's Hospital}}
        {\textbf{Boston, MA}}
    \\
    \headerrow
        {\emph{Research Assistant}}
        {\emph{Summer 2012}}

        \begin{itemize*}
        \item Conducted research at the Center for Clinical Spectroscopy 
        \item Developed a computer program in Felix Command Line (a proprietary language) to do post-processing of magnetic resonance spectroscopy scans, improving efficiency by >50\%
        \end{itemize*}

\end{itemize}




\hrule
\vspace{-0.4em}
\subsection*{Core Technical Skills}

\begin{indentsection}{\parindent}
\hyphenpenalty=1000
\begin{description*}
	\item[Languages:]
	C, Java, SQL, SVG
    \item[Tools:]
    Eclipse, Git, Unix, Vim
\end{description*}
\end{indentsection}

\end{document}
